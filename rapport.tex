
\documentclass[a4paper,11pt]{report}
\usepackage[utf8]{inputenc}
\usepackage{color,amsmath,xcolor,listings,graphicx}
\usepackage[francais]{babel}

% paramétrage pour les zones de code perl
\lstset{
    language=Perl, commentstyle=\textit, frame=shadowbox,
    rulesepcolor=\color{gray}, basicstyle=\ttfamily\small, columns=flexible,
    tabsize=3, extendedchars=true, showspaces=false,
    showstringspaces=false, numbers=left, numberstyle=\tiny,
    breaklines=true, breakautoindent=true, captionpos=b, morecomment=[l]{//}
}

%language=Octave %-> choose the language of the code
%basicstyle=\footnotesize %-> the size of the fonts used for the code
%numbers=left %-> where to put the line-numbers
%numberstyle=\footnotesize %-> size of the fonts used for the line-numbers
%stepnumber=2 -> the step between two line-numbers.
%numbersep=5pt -> how far the line-numbers are from the code
%backgroundcolor=\color{white} -> sets background color (needs package)
%showspaces=false -> show spaces adding particular underscores
%showstringspaces=false -> underline spaces within strings
%showtabs=false -> show tabs within strings through particular underscores
%frame=single -> adds a frame around the code
%tabsize=2 -> sets default tab-size to 2 spaces
%captionpos=b -> sets the caption-position to bottom
%breaklines=true -> sets automatic line breaking
%breakatwhitespace=false -> automatic breaks happen at whitespace
%morecomment=[l]{//} -> displays comments in italics (language dependent)


% infos du document
\title{QNAP}
\author{Boulic Guillaume, Émeric Tosi}
\date{\today}


%
\begin{document}

    \maketitle{} % Afficher la page de garde : Titre + Auteur(s) + Date de dernière compilation


%    \begin{figure} % on s'en fout de l'image moche, c'est juste pour test xD
%        \begin{center}
            %\includegraphics{network.png}
            %\includegraphics[height=128, width=128]{network.png}
            %\includegraphics[scale=0.5]{network.png}
%        \end{center}
%            \caption{ Laule } % ce qui apparait juste en dessous de l'image
%            \label{c'est styler !}
%    \end{figure}

    \setcounter{tocdepth}{1} % définir la profondeur de l'Index
    \renewcommand{\contentsname}{Sommaire} % renommer l'Index en Sommaire
    \tableofcontents{} % afficher l'Index
    \clearpage


% Differentes Parties / Chapitres / Autres fichiers à inclure :

\chapter*{Introduction}
\addcontentsline{toc}{chapter}{Introduction}
        \paragraph{}
Blablabla.
        \paragraph{}
Lorem ipsum dolor sit amet, consectetur adipisicing elit, sed doeiusmod tempor incididunt ut labore et dolore magna aliqua.
Ut enimad minim veniam, quis nostrud exercitation ullamco laboris nisi utaliquip ex ea commodo consequat.
Duis aute irure dolor inreprehenderit in voluptate velit esse cillum dolore eu fugiat nullapariatur.
Excepteur sint occaecat cupidatat non proident, sunt inculpa qui officia deserunt mollit anim id est laborum.
    \clearpage


%

\chapter{Code de Répétition}

    \section{Introduction}

        \paragraph{}
On transmet simplement plusieurs fois le même message, on double les bits à transmettre pour recevoir deux fois le même message,
si les messages reçus ne concordent pas, alors il y a eu une erreur dans la transmission.
On peut choisir deux méthodes pour doubler les bits, doubler chaque bit ou doubler le message complet.
\\ Exemple :
\\ 101 deviens 110011 si on double chaque bit
\\ 101 deviens 101101 si on double le message


    \clearpage

    \section{Fiabilité}

        \paragraph{}
Une seule erreur peut être détectée à coup sur.
Les erreurs ne sont pas detectés dans les cas d'un nombre d'erreur pair se situant au même bits dans les deux messages envoyés.


    \section{Probabilité de détection}

        \paragraph{}
Nombre d'erreurs qui ne sont pas détectées (par dénombrement des cas d'indection cités précédement):
\[  \text{Nombre d'erreurs indétectables} = 123 \]
Nombre de cas totaux :
\[  \text{Nombre de cas} = 2^{(7*2)} \]
Probabilité de détection :
\[  P(\text{Détection}) = \frac{\text{Nombre de cas} - \text{Nombre d'erreurs indétectables}}{\text{Nombre de cas}}*100 \]
On obtient alors :
\[  P(\text{Détection}) = \frac{2^{(7*2)} - 123}{2^{(7*2)}} \]
\[  P(\text{Détection}) \approx 99.25\% \]


    \section{Rendement}

        \paragraph{}
Le rendement de ce code est très mauvais, on double au minimum la taille du message :
\[  Rendement = \text{Taille du message}*\text{Nombre de répétitions} \]
Pour un message sur 7 bits (un simple caractère encodé en UTF-7 par exemple) et avec 1 répétition seulement :
\[  Rendement = \frac{7}{7*2} = 50\% \]
Pour le même message mais avec 2 répétitions :
\[  Rendement = \frac{7}{7*3} \approx 33.3\% \]


    \clearpage
    \section{Exemple pratique}

        \lstset{
            language=bash, basicstyle=\ttfamily\small, columns=flexible,
            tabsize=2, extendedchars=true, showspaces=false,
            showstringspaces=false, numbers=left, numberstyle=\tiny,
            breaklines=true, breakautoindent=true, captionpos=b
        }
Voir détail du code en annexe.
        \begin{lstlisting}
>> perl envoiNoise.pl 8000 127.0.0.1:9000

 -> envoi du caractere : o

 -> nombre de caracteres envoyes : 100000

        \end{lstlisting}

        \begin{lstlisting}

>> perl receptionNoise.pl 9000 127.0.0.1:8000

 -> nombre de receptions : 100000
 -> nombre de receptions supposees bonnes : 18867
 -> nombre de vrais bons caracteres : 18255
 -> nombre d'erreurs au total : 81745
 -> nombre d'erreurs detectees : 81133
 -> nombre d'erreurs non detectees : 612
 -> fiabilitee de l'envoi/reception : 18.255%
 -> fiabilitee de detection d'erreur : 99.2513303565967%

        \end{lstlisting}

\clearpage


%

\chapter{VRC : Bit de parité}

    \section{Introduction}

        \paragraph{}
le VRC (Vertical Redundancy Check), plus connu sous le nom de bit de parité,
est simplement le rajout d'un bit en fin de message pour assurer la parité du message.
Ce dernier bit la valeur nécessaire pour assurer un nombre pair de bit à 1 dans le message final.
Il est donc à 0 pour un nombre pair de bit à 1 dans le message de départ, ou est à 1.


    \clearpage

    \section{Fiabilité}

        \paragraph{}
Une seule erreur peut être détectée à coup sur.
Toutes les erreurs où un nombre pair de bits a été modifié ne sont pas détectées,
les erreurs détectées sont donc celles où un nombre impair de bits a changé d'état.
Si une seule erreur intervient mais porte sur le bit de parité, le message est considéré comme invalide.
Ce code ne permet pas la correction d'erreur, il est nécessaire de demander à nouveau l'envoi du message détecté invalide.


    \section{Probabilité de détection}

        \paragraph{}
\[  P(\text{Transmission Parfaite}) = P(X=0) = {8\choose 0}p^0(1-p)^{8-0} = (1-p)^8 \]
        \paragraph{}
\[  P(\text{Message Erroné}) = 1 - P(\text{Transmission Parfaite}) = 1 - (1-p)^8 \]
        \paragraph{}
\[  P(\text{Détection}) = P(\text{1 erreur}) + P(\text{3 erreurs}) + P(\text{5 erreurs}) + P(\text{7 erreurs}) \]
\[ = {8\choose 1}p^1(1-p)^{8-1} + {8\choose 3}p^3(1-p)^{8-3} + {8\choose 5}p^5(1-p)^{8-5} + {8\choose 7}p^7(1-p)^{8-7} \]
\[ = 8p(1-p)^7 + {8\choose 3}p^3(1-p)^5 + {8\choose 5}p^5(1-p)^3 + 8p^7(1-p) \]
        \paragraph{}
\[  P(\text{Reconnaissance Erreur}) = \frac{P(\text{Détection})}{P(\text{Message Erroné})}\]
        \paragraph{}
Pour une probabilité de 10\% d'erreurs :
\[  P(\text{Transmission Parfaite}) = (1-0.1)^{8} = 43\%\]
\[  P(\text{Message Erroné}) = 1 - 0.43 = 57\% \]
\[  P(\text{Détection}) = 8*0.1(1-0.1)^7 + {8\choose 3}0.1^3(1-0.1)^5 + {8\choose 5}0.1^5(1-0.1)^3 + 8*0.1^7(1-0.1) \]
\[  P(\text{Détection}) = 0.8*0.9^7 + 56*0.1^3*0.9^5 + 56*0.1^5*0.9^3 + 8*0.1^7*0.9 \]
\[  P(\text{Détection}) = 42\% \]
\[  P(\text{Reconnaissance Erreur}) = \frac{0.42}{0.57} \approx 74\% \]


    \section{Rendement}

        \paragraph{}
Le rendement de ce code est très bon :
\[  Rendement = \frac{\text{Taille du message}}{\text{Taille du message}+1} \]
        \paragraph{}
Pour notre message d'exemple (un simple caractère encodé en UTF-7) le rendement est déjà excellent :
\[  Rendement = \frac{7}{7+1} = 87.5\% \]


    \clearpage
    \section{Exemple pratique}

        \lstset{
            language=bash, basicstyle=\ttfamily\small, columns=flexible,
            tabsize=2, extendedchars=true, showspaces=false,
            showstringspaces=false, numbers=left, numberstyle=\tiny,
            breaklines=true, breakautoindent=true, captionpos=b
        }
Voir détail du code en annexe.
        \begin{lstlisting}
>> perl envoiNoise.pl 9001 127.0.0.1:9000

 -> envoi du caractere : o

 -> nombre de caracteres envoyes : 100000

        \end{lstlisting}

        \begin{lstlisting}
>> perl receptionNoise.pl 9000 127.0.0.1:9001

 -> nombre de receptions : 100000
 -> nombre de receptions supposees bonnes : 58311
 -> nombre de vrais bons caracteres : 43079
 -> nombre d'erreurs au total : 56921
 -> nombre d'erreurs detectees : 41689
 -> nombre d'erreurs non detectees : 15232
 -> fiabilitee de l'envoi/reception : 43.079%
 -> fiabilitee de detection d'erreur : 73.2401047065231%

        \end{lstlisting}

\clearpage


%

\chapter{LRC : Contrôle parité croisée}

    \section{Introduction}

        \paragraph{}
Le LRC (Longitudinal Redundancy Check) est un mot d'information se composant $L$ caractere de texte.
Il s'agit d'un double codage par bits de parité. Les 7 $L$ bits sont rangés dans un tableau de $L$ lignes et 7 colonnes.
Chaque ligne est complété par un bit de parité et de même pour les 8 colonnes formés.
        \paragraph{}
Pour rester cohérent avec notre message d'exemple nous prendrons le cas d'une matrice carrée composée de 7 messages de taille 7 chacun.
Il sera appliqué horizontalement à la matrice (à chaque message) le bit de parité.
La matrice passera donc à une taille de (7+1) sur 8.
Le dernier message qui sera généré grâce à l'application du bit de parité verticalement sur la matrice.


    \clearpage

    \section{Fiabilité}

        \paragraph{}
%cb erreur peut être corrigée. ?
1 erreur peut être corrigée et jusqu’à trois erreurs peuvent être détectées à coup sûres.


    \section{Probabilité de détection}

        \paragraph{}
Probabilité d'exactitude d'un message :
\[  P(\text{Exact}) = P(\text{0 erreur}) + P(\text{1 erreur}) \]
\[ \text{Taille de la matrice } n = ( \text{ Nombre de lignes } + 1 ) * ( \text{ Longeur du message } + \text{ Bit de parité } ) \]
\[ P(\text{0 erreur}) + P(\text{1 erreur}) = (1-p)^{n} + {n\choose 1}*p*(1-p)^{n-1} \]
Dans notre cas et avec les 10\% de chance qu'un bit soit changé :
\[ \text{Taille de la matrice } n = 8*(7+1) = 8^{2} \]
\[ \text{Taille de la matrice } n = 64 \]
\[ P(\text{0 erreur}) + P(\text{1 erreur}) = (1-p)^{64} + 64*p*(1-p)^{63} \]
\[ P(\text{0 erreur}) + P(\text{1 erreur}) = (1-0.1)^{64} + 64*0.1*(1-0.1)^{63} \]
\[ P(\text{0 erreur}) + P(\text{1 erreur}) = (0.9)^{64} + 64*0.1*(0.9)^{63} \]
\[ P(\text{0 erreur}) + P(\text{1 erreur}) \approx 0.1179\% \]
Un message a donc environ 0.1179\% de chance d'être décodé correctement, c'est à dire non détecté faux.
Le nombre de cas d'erreur où le LRC est mis à défaut est :
\[  \text{Nombre d'indetections} = (49+42+35+28+21+14+7)*7 \]
le nombre de cas total est  :
\[  P(\text{Nomber de cas}) = 2^{8*(7+1)} = 2^{8*8} \]
\[  P(\text{Nombre de cas}) = 2^{64} \]
\[  P(\text{Détection}) = \frac{\text{Nombre de cas} - \text{Nombre d'indetections}}{\text{Nombre de cas}}*100 \]
\[  P(\text{Détection}) \approx 100\% \]
Comparons la probabilité d'un message decodé à l'aide du LRC à celle de l'information non codé.
L'information esr alors transmis directement avec la probabilité de transmission sans erreur.
On a :
\[  P(\text{n'}) = 7L=7*8=56 \]
\[  P(\text{0}) = (1-p)^{n'}=\frac{9}{10} = 0.00273927\]
Calculons alors l'amélioration :
\[  P(\text{Amelioration}) = \frac{P(\text{Exact}) - P(\text{0})}{P(\text{0})} \]
\[ =\frac{(\frac{9}{10})^{63}*(\frac{64}{10})-(\frac{9}{10})^{56}}{(\frac{9}{10})^{56}} \]
\[ =(\frac{9}{10}^{7}*\frac{64}{10})-1 \approx 2.0611 \]
On obtient donc une amélioration de 206,11\%.
    \section{Rendement}

        \paragraph{}
Le rendement de ce code est bon :
\[  Rendement = \frac{n*\text{Taille du message}}{(n+1)*(\text{Taille du message}+1)} \]
        \paragraph{}
Dans le cas que nous étudions (7 messages de 7 bits):
\[  Rendement = \frac{7*7}{(7+1)*(7+1)} \approx 76.6\% \]

\clearpage


%
\chapter*{Conclusion}
\addcontentsline{toc}{chapter}{Conclusion}
        \paragraph{}
Too much bullshit here :P
    \clearpage


%
\chapter*{Résumé}
\addcontentsline{toc}{chapter}{Résumé}
\paragraph{}
Blablabla ...
    \clearpage


%
\chapter*{Abstract}
\addcontentsline{toc}{chapter}{Abstract}
\paragraph{}
Blblblbl ...
    \clearpage


%

\appendix{}

\chapter{Annexes}

    \section{Exemple(s)}

        \paragraph{}
            \emph{emphatique}
            \textbf{gras}
            \texttt{machine à écrire}
            \textsl{incliné}
            \textsc{Petites majuscules}

        \paragraph{}
            The foundations of the rigorous study of \emph{analysis}
            were laid in the nineteenth century, notably by the
            mathematicians Cauchy and Weierstrass. Central to the
            study of this subject are the formal definitions of
            \emph{limits} and \emph{continuity}.

        \paragraph{}
            Let $D$ be a subset of $\bf R$ and let
            $f \colon D \to \mathbf{R}$ be a real-valued function on
            $D$. The function $f$ is said to be \emph{continuous} on
            $D$ if, for all $\epsilon > 0$ and for all $x \in D$,
            there exists some $\delta > 0$ (which may depend on $x$)
            such that if $y \in D$ satisfies
            \[ |y - x| < \delta \]
            then
            \[ |f(y) - f(x)| < \epsilon. \]

        \paragraph{}
            One may readily verify that if $f$ and $g$ are continuous
            functions on $D$ then the functions $f+g$, $f-g$ and
            $f.g$ are continuous. If in addition $g$ is everywhere
            non-zero then $f/g$ is continuous.

    \clearpage


%
    \section{Répétition}
%
        \paragraph{}
Implémentation de l'envoi pour le Code de Répétition
\lstinputlisting{./Repetition/Perl/receptionNoise.pl}
    \clearpage
%
        \paragraph{}
Implémentation de la réception pour le Code de Répétition
\lstinputlisting{./Repetition/Perl/envoiNoise.pl}
    \clearpage
%
        \paragraph{}
Implémentation de la vérification pour le Code de Répétition
\lstinputlisting{./Repetition/Perl/Repetition.pm}
    \clearpage


%
    \section{VRC}
%
        \paragraph{}
Implémentation de l'envoi pour le VRC
\lstinputlisting{./VRC/Perl/receptionNoise.pl}
    \clearpage
%
        \paragraph{}
Implémentation de la réception pour le VRC
\lstinputlisting{./VRC/Perl/envoiNoise.pl}
    \clearpage
%
        \paragraph{}
Implémentation de la vérification VRC
\lstinputlisting{./VRC/Perl/Parity.pm}
    \clearpage

\clearpage


%
%\listoffigures % index des images du rapport
%\clearpage


% fin
\end{document}
