%
\chapter{Commutation de paquets}
%
    \section{Un commutateur de paquets}
%
        \subsection{Énoncé}
%
            \paragraph{}
Nous cherchons à simuler un lien de sortie d'un commutateur de paquets.
            %    \begin{figure}
            %        \begin{center}
                        %\includegraphics[scale=0.5]{RSC/2.1.png}
            %        \end{center}
            %            \caption{ Schéma de fonctionnement d'un commutateur de paquets }
            %            \label{ Schéma de fonctionnement d'un commutateur de paquets }
            %    \end{figure}
%
            \paragraph{}
L'arrivée des paquets est supposée suivre une loi exponentielle de paramètre $\lambda$.
Nous positionnons une file en sortie du commutateur pour stocker les différents paquets.
Les paquets ont une longueur exponentielle-ment distribuée de paramètre $\frac{1}{\nu} = 10 000 bits$.
Le lien de sortie a un débit de 10 Mbit/s.
%
        \subsection{Calcul analytique du temps moyen de service $\frac{1}{\mu}$}
\[  \text{Temps moyen de service} = \frac{1}{\mu} \]
\[ \iff \frac{1}{\nu} \frac{1}{D} = 10^{4} \frac{1}{10^{7}} = \frac{1}{10^{3}} = 10^{-3} \ \text{seconde} \]
%
        \subsection{Déterminer le nombre moyen de paquets dans la file et le temps moyen de réponse en fonction du taux d'arrivée pour différentes durées de simulation}
\[  \lambda = \rho \mu \]
\[  \text{Charge de trafic} \ \rho = \frac{\lambda}{\mu} \]
\[  \text{Nombre moyen de client} \ \bar{N} = \frac{\rho}{(1 - \rho)} \]
\[  \text{Temps moyen de reponse} \ \bar{W} = \frac{1}{(\mu - \lambda)} \]
\[  \bar{N} = \lambda \bar{W} \]
\begin{center}
    \begin{tabular}{ | c | c| c | c | }
        \hline
            $\rho$ & 0.1 & 0.5 & 0.9 \\
        \hline
            $\lambda$ & $10^{2}$ & $5 10^{2}$ & $9 10^{2}$ \\
        \hline
            Nombre moyen de client & 0.11111 & 1 & 9 \\
        \hline
            Temps moyen de réponse & 0.00111 & 0.002 & 0.01 \\
        \hline
    \end{tabular}
\end{center}
%
        \subsection{Comparaison du résultat de la simulation avec la théorie}
            \paragraph{}
Blablabla 1
    % TODO >>>
    % SIMULATION
    % ici graphique
    % TODO <<<
\begin {figure}
    \centering
        % GNUPLOT: LaTeX picture with Postscript
\begingroup
  \makeatletter
  \providecommand\color[2][]{%
    \GenericError{(gnuplot) \space\space\space\@spaces}{%
      Package color not loaded in conjunction with
      terminal option `colourtext'%
    }{See the gnuplot documentation for explanation.%
    }{Either use 'blacktext' in gnuplot or load the package
      color.sty in LaTeX.}%
    \renewcommand\color[2][]{}%
  }%
  \providecommand\includegraphics[2][]{%
    \GenericError{(gnuplot) \space\space\space\@spaces}{%
      Package graphicx or graphics not loaded%
    }{See the gnuplot documentation for explanation.%
    }{The gnuplot epslatex terminal needs graphicx.sty or graphics.sty.}%
    \renewcommand\includegraphics[2][]{}%
  }%
  \providecommand\rotatebox[2]{#2}%
  \@ifundefined{ifGPcolor}{%
    \newif\ifGPcolor
    \GPcolorfalse
  }{}%
  \@ifundefined{ifGPblacktext}{%
    \newif\ifGPblacktext
    \GPblacktexttrue
  }{}%
  % define a \g@addto@macro without @ in the name:
  \let\gplgaddtomacro\g@addto@macro
  % define empty templates for all commands taking text:
  \gdef\gplbacktext{}%
  \gdef\gplfronttext{}%
  \makeatother
  \ifGPblacktext
    % no textcolor at all
    \def\colorrgb#1{}%
    \def\colorgray#1{}%
  \else
    % gray or color?
    \ifGPcolor
      \def\colorrgb#1{\color[rgb]{#1}}%
      \def\colorgray#1{\color[gray]{#1}}%
      \expandafter\def\csname LTw\endcsname{\color{white}}%
      \expandafter\def\csname LTb\endcsname{\color{black}}%
      \expandafter\def\csname LTa\endcsname{\color{black}}%
      \expandafter\def\csname LT0\endcsname{\color[rgb]{1,0,0}}%
      \expandafter\def\csname LT1\endcsname{\color[rgb]{0,1,0}}%
      \expandafter\def\csname LT2\endcsname{\color[rgb]{0,0,1}}%
      \expandafter\def\csname LT3\endcsname{\color[rgb]{1,0,1}}%
      \expandafter\def\csname LT4\endcsname{\color[rgb]{0,1,1}}%
      \expandafter\def\csname LT5\endcsname{\color[rgb]{1,1,0}}%
      \expandafter\def\csname LT6\endcsname{\color[rgb]{0,0,0}}%
      \expandafter\def\csname LT7\endcsname{\color[rgb]{1,0.3,0}}%
      \expandafter\def\csname LT8\endcsname{\color[rgb]{0.5,0.5,0.5}}%
    \else
      % gray
      \def\colorrgb#1{\color{black}}%
      \def\colorgray#1{\color[gray]{#1}}%
      \expandafter\def\csname LTw\endcsname{\color{white}}%
      \expandafter\def\csname LTb\endcsname{\color{black}}%
      \expandafter\def\csname LTa\endcsname{\color{black}}%
      \expandafter\def\csname LT0\endcsname{\color{black}}%
      \expandafter\def\csname LT1\endcsname{\color{black}}%
      \expandafter\def\csname LT2\endcsname{\color{black}}%
      \expandafter\def\csname LT3\endcsname{\color{black}}%
      \expandafter\def\csname LT4\endcsname{\color{black}}%
      \expandafter\def\csname LT5\endcsname{\color{black}}%
      \expandafter\def\csname LT6\endcsname{\color{black}}%
      \expandafter\def\csname LT7\endcsname{\color{black}}%
      \expandafter\def\csname LT8\endcsname{\color{black}}%
    \fi
  \fi
    \setlength{\unitlength}{0.0500bp}%
    \ifx\gptboxheight\undefined%
      \newlength{\gptboxheight}%
      \newlength{\gptboxwidth}%
      \newsavebox{\gptboxtext}%
    \fi%
    \setlength{\fboxrule}{0.5pt}%
    \setlength{\fboxsep}{1pt}%
\begin{picture}(7200.00,5040.00)%
    \gplgaddtomacro\gplbacktext{%
      \csname LTb\endcsname%
      \put(594,440){\makebox(0,0)[r]{\strut{}$0$}}%
      \put(594,1059){\makebox(0,0)[r]{\strut{}$0.1$}}%
      \put(594,1679){\makebox(0,0)[r]{\strut{}$0.2$}}%
      \put(594,2298){\makebox(0,0)[r]{\strut{}$0.3$}}%
      \put(594,2917){\makebox(0,0)[r]{\strut{}$0.4$}}%
      \put(594,3536){\makebox(0,0)[r]{\strut{}$0.5$}}%
      \put(594,4156){\makebox(0,0)[r]{\strut{}$0.6$}}%
      \put(594,4775){\makebox(0,0)[r]{\strut{}$0.7$}}%
      \put(726,220){\makebox(0,0){\strut{}$0$}}%
      \put(1334,220){\makebox(0,0){\strut{}$10$}}%
      \put(1941,220){\makebox(0,0){\strut{}$20$}}%
      \put(2549,220){\makebox(0,0){\strut{}$30$}}%
      \put(3157,220){\makebox(0,0){\strut{}$40$}}%
      \put(3765,220){\makebox(0,0){\strut{}$50$}}%
      \put(4372,220){\makebox(0,0){\strut{}$60$}}%
      \put(4980,220){\makebox(0,0){\strut{}$70$}}%
      \put(5588,220){\makebox(0,0){\strut{}$80$}}%
      \put(6195,220){\makebox(0,0){\strut{}$90$}}%
      \put(6803,220){\makebox(0,0){\strut{}$100$}}%
    }%
    \gplgaddtomacro\gplfronttext{%
      \csname LTb\endcsname%
      \put(5816,4602){\makebox(0,0)[r]{\strut{}minT}}%
      \csname LTb\endcsname%
      \put(5816,4382){\makebox(0,0)[r]{\strut{}maxT}}%
      \csname LTb\endcsname%
      \put(5816,4162){\makebox(0,0)[r]{\strut{}moyenne temps traitement}}%
      \csname LTb\endcsname%
      \put(5816,3942){\makebox(0,0)[r]{\strut{}minA}}%
      \csname LTb\endcsname%
      \put(5816,3722){\makebox(0,0)[r]{\strut{}maxA}}%
      \csname LTb\endcsname%
      \put(5816,3502){\makebox(0,0)[r]{\strut{}mayyenne nb packet att}}%
    }%
    \gplbacktext
    \put(0,0){\includegraphics{p2.q3}}%
    \gplfronttext
  \end{picture}%
\endgroup

    \centering
\end {figure}
%
        \subsection{Cas où les paquets ont une longueur constante (10000 bits)}
%
            \subsubsection{Calculer analytiquement le temps moyen de service $\frac{1}{\mu}$}
%
                \paragraph{}
On obtient la même chose que précédemment :
\[  \text{Temps moyen de service} = \frac{1}{\mu} \]
\[ \iff \frac{1}{\nu} \frac{1}{D} = 10^{4} \frac{1}{10^{7}} = \frac{1}{10^{3}} = 10^{-3} \ \text{seconde} \]
%
            \subsubsection{Résultats en fonction du taux d'arrivée pour différentes durées de simulation}
%
% voir pour trouver des formules, la progression doit être linéaire en fonction de $\rho$.
%
                \paragraph{Temps moyen de réponse}
Blablabla 3
%
                \paragraph{Nombre moyen de paquets dans la file d'attente}
Blablabla 4
%
            \subsubsection{Analyse et comparaison des résultats}
%
                \paragraph{}
Blablabla 5
    % TODO >>>
    % SIMULATION
    % ici graphique
    % TODO <<<
%
    \clearpage
%
